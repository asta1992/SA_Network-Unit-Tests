\documentclass[a4,12pt]{scrartcl}

%Basic 
\usepackage[utf8]{inputenc}
\usepackage[ngerman]{babel}
\usepackage[T1]{fontenc}
%Schrift 
%\usepackage{fontspec} 
%\setmainfont{Arial} 
%Zeilenabstand
\usepackage{setspace}
\setstretch {1.3}
\usepackage{float}
\usepackage[bottom = 3.50cm]{geometry}

%Titel Seite
\usepackage{titling} %Wird benötigt damit \maketitle die Variabeln title, author und date nicht überschreibt
\title{Test Cases}
%\subtitle{Projekt: software name}
\author{David Meister \and Andreas Stalder}		
 %mit /and können Personen hinzugefügt werden
\date{\today}


%Kopf, Fusszeile
\usepackage{fancyhdr}
\pagestyle{fancy}
\lhead{}
\chead{}
\rhead{Architektur}
\lfoot{\thetitle \: v1.0 }
\cfoot{\today}
\rfoot{Seite \thepage}
\renewcommand{\headrulewidth}{0.4pt}

%Bilder
\usepackage{graphicx}

%Zeichnen
\usepackage{tikz}

%Tabellen
\usepackage{booktabs}
\usepackage{longtable}

%Codesnippets
\usepackage{listings}
\lstset{language=java,basicstyle=\footnotesize,frame=single} %backgroundcolor=\color{lightgray}

%Querformat für eine Seite
\usepackage{lscape}
\usepackage{rotating}
\usepackage{pdflscape}

%URL 
\usepackage[colorlinks=true, linkcolor=blue, urlcolor=blue, citecolor=blue]{hyperref}
\urlstyle{same} 


%Loremimpsum
\usepackage{lipsum}



\begin{document}

%\clearpage\maketitle
\begin{titlepage}
	\centering
	\vspace{5cm}
	\begin{center}
%	\includegraphics[width=0.50\textwidth]{}
	\end{center}
%	{\huge\bfseries software name\par}
	\vspace{8cm}
	\raggedright
	{\bfseries HSR Studienarbeit Network Unit Testing\par}
	{\huge\bfseries Architektur \par}
	\vspace{1cm}
	{\theauthor \par}
	{\today\par}

\end{titlepage}

\section{Änderungsgeschichte}

\begin{table}[htb]
\centering
    \begin{tabular}{@{} l l l l@{}}\toprule    
    {Datum} & {Version} & {Änderung} & {Autor}\\ \midrule
    1.11.16 & 1.0 & Erstellung erster Version & dm/as\\ \addlinespace
    \end{tabular}
\caption{\textbf{Änderungsgeschichte}}
\end{table}

\newpage

%\thispagestyle{empty}
\tableofcontents
\newpage


\section{Einführung}
\subsection{Zweck}
Todo Architektur Zweck
\subsection{Gültigkeitsbereich}
Dieses Dokument ist über die gesamte Projektdauer gültig. Es wird in späteren Iterationen angepasst. Somit ist jeweils die neuste Version des Dokuments gültig und alte Versionen sind obsolet.
\subsection{Referenzen}
\begin{description}
Todo Ref zu Automation Plattform und anderen Dokumenten
\end{description}
\newpage
\section{Systemübersicht}
\newpage
\section{Klassenstruktur}
\newpage
\section{Logische Architektur}
\begin{figure} [H]
	\begin{center}
	\includegraphics[width=0.90\textwidth]{./pictures/architektur.png}
	\label{Bild Referenz}
	\end{center}
\end{figure}
\begin{figure} [H]
	\begin{center}
	\includegraphics[width=0.90\textwidth]{./pictures/dependence_diagram.png}
	\label{Bild Referenz}
	\end{center}
\end{figure}


\subsection{Application-Layer}
Im Application-Layer befinden sich alle Abläufe, um durch das Programm zu führen. Das heisst, dass jeder Parameter seine eigene Controller-Klasse erhalten wird. Da momentan nur zwei Parameter (-i und -v) vorhanden sind, gibt es auch nur zwei Controller.
\subsubsection{Klassenstruktur}
\begin{table}[H]
\centering
    \begin{tabular}{@{}l p{11cm} @{}}\toprule    
    {Klassenname} & {Beschreibung}\\ \midrule
    ValidatorController & Der ValidatorController beinhaltet die Logik und Ausgaben, um durch die Validierung einer Datei zu führen.\\       
    TestController & Der TestController beinhaltet die gesamte Logik um Tests zu erstellen, auszuführen und zu testen. Dazu verwendet er Teile aus dem Service-Layer. \\
    \bottomrule
    \end{tabular}
\end{table}
\subsubsection{Schnittstellen}
Der Application-Layer hat momentan nur eine Schnittstelle in den Service-Layer. So kann er die verschiedenen Aufgaben an die untere schickt weiterleiten.



\subsection{Service-Layer}
Der Service-Layer beinhaltete die gesamte Programmlogik. Das heisst es gibt vier verschiedene Klassen, - Runner, FileHandler, Evaluator, FileValidator - welche jeweils ein Teil der gesamten Aufgabe übernehmen.
\subsubsection{Klassenstruktur}
\begin{table}[H]
\centering
    \begin{tabular}{@{}l p{11cm} @{}}\toprule    
    {Klassenname} & {Beschreibung}\\ \midrule
    
    Runner & Alle erfassten Tests werden durch den Runner von der TestSuite geholt und es wird ein CMD für Salt zusammengestellt. Dieses wird dann ausgeführt und die Testresultate werden der TestSuite zurückgegeben.  \\       
    Evaluator & Der Evaluator vergleicht die erwarteten Werte aus der TestSuite mit den tatsächlichen Resultaten. \\
    FileHandler & Mit dem FileHandler werden die Testfiles und die Inventoryfiles eingelesen und in Objekte umgewandelt.\\
    FileValidator & Der FileValidator überprüft ein Testfile und gibt zurück, ob dieses korrekt formatiert ist und ob die eingegebenen Parameter gültig sind. \\

    \bottomrule
    \end{tabular}
\end{table}
\subsubsection{Schnittstellen}
Der Service-Layer hat eine Schnittstelle zum Data-Layer. Durch den Service-Layer werden die Datenobjekte erstellt, bearbeitet und ausgewertet. Obere schickten müssen so immer über den Service-Layer, um eine saubere Abtrennung der Schichten zu gewährleisten.



\subsection{Data-Layer}
Der Data-Layer beinhaltet alle Datenobjekte, welche vom laufenden Programm benötigt werden.
\subsubsection{Klassenstruktur}
\begin{table}[H]
\centering
    \begin{tabular}{@{}l p{11cm} @{}}\toprule    
    {Klassenname} & {Beschreibung}\\ \midrule
    
    Testsuite & Die Testsuite beinhaltet die Logik, um neue Testcases und Devices anzulegen und abfragen auf die Daten zu machen.\\     
    TestCase & In der Klasse TestCase, werden alle Daten zu einem TestCase gespeichert. Diese kommen vom Testfile, welches der User vorgängig erfasst hat.\\         
    Devices & Alle Daten zu den einzelnen Devices aus dem Inventoryfile, werden hier als Objekt abgespeichert.\\
  
    \bottomrule
    \end{tabular}
\end{table}
\subsubsection{Schnittstellen}
Da der Data-Layer der unterste Layer ist, gibt es hier keine weiteren Schnittstellen.







\newpage
\section{Datensicherung}
Nuts speichert nur sehr wenige Daten. Dazu gehört ein Result-Log pro Ausführung und ein Error-Log für die Fehler.
\subsection{Testdateien}
Die erstellten Testdateien müssen von jedem Benutzer selber abgespeichert und verwaltet werden. Nuts speichert diese selber nicht ab.
\subsection{Auswertungen}
Jeder ausgeführte Test ergibt eine Log-Datei. Dieser Testlog wird unter /var/log/nuts/ gespeichert. Die Datei beinhaltet die Testergebnisse und weitere Details, falls der Test nicht bestanden ist.
\subsection{Error-Log}
Falls es beim ausführen des Programms zu einem Fehler kommt, wird dieser Fehler im Error-Log erfasst. Der Error-Log wird unter /var/log/nuts/error.log abgespeichert.

\end{document}

