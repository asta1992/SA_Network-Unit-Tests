\documentclass[a4,12pt]{scrartcl}

%Basic 
\usepackage[utf8]{inputenc}
\usepackage[ngerman]{babel}
\usepackage[T1]{fontenc}
%Schrift 
%\usepackage{fontspec} 
%\setmainfont{Arial} 
%Zeilenabstand
\usepackage{setspace}
\setstretch {1.3}
\usepackage{float}
\usepackage[bottom = 3.50cm]{geometry}

%Titel Seite
\usepackage{titling} %Wird benötigt damit \maketitle die Variabeln title, author und date nicht überschreibt
\title{Test Cases}
%\subtitle{Projekt: software name}
\author{David Meister \and Andreas Stalder}		
 %mit /and können Personen hinzugefügt werden
\date{\today}


%Kopf, Fusszeile
\usepackage{fancyhdr}
\pagestyle{fancy}
\lhead{}
\chead{}
\rhead{Network Testing Detailed}
\lfoot{\thetitle \: v1.0 }
\cfoot{\today}
\rfoot{Seite \thepage}
\renewcommand{\headrulewidth}{0.4pt}

%Bilder
\usepackage{graphicx}

%Zeichnen
\usepackage{tikz}

%Tabellen
\usepackage{booktabs}
\usepackage{longtable}

%Codesnippets
\usepackage{listings}
\lstset{language=java,basicstyle=\footnotesize,frame=single} %backgroundcolor=\color{lightgray}

%Querformat für eine Seite
\usepackage{lscape}
\usepackage{rotating}
\usepackage{pdflscape}

%URL 
\usepackage[colorlinks=true, linkcolor=blue, urlcolor=blue, citecolor=blue]{hyperref}
\urlstyle{same} 


%Loremimpsum
\usepackage{lipsum}



\begin{document}

%\clearpage\maketitle
\begin{titlepage}
	\centering
	\vspace{5cm}
	\begin{center}
%	\includegraphics[width=0.50\textwidth]{}
	\end{center}
%	{\huge\bfseries software name\par}
	\vspace{8cm}
	\raggedright
	{\bfseries HSR Studienarbeit Network Unit Testing\par}
	{\huge\bfseries Network Testing Detailed\par}
	\vspace{1cm}
	{\theauthor \par}
	{\today\par}

\end{titlepage}

\section{Änderungsgeschichte}

\begin{table}[htb]
\centering
    \begin{tabular}{@{} l l l l@{}}\toprule    
    {Datum} & {Version} & {Änderung} & {Autor}\\ \midrule
    10.10.16 & 1.0 & Erstellung erster Version & dm/as\\ \addlinespace
    \end{tabular}
\caption{\textbf{Änderungsgeschichte}}
\end{table}

\newpage

%\thispagestyle{empty}
\tableofcontents
\newpage


\section{Einführung}
\subsection{Zweck}
Dieses Dokument beschreibt die Anforderungen mittels Use Cases und nichtfunktionalen
Anforderungen.
\subsection{Gültigkeitsbereich}
Dieses Dokument ist über die gesamte Projektdauer gültig. Es wird in späteren Iterationen angepasst. Somit ist jeweils die neuste Version des Dokuments gültig und alte Versionen sind obsolet.
\subsection{Referenzen}

\newpage
\section{Einleitung}

\newpage
\section{Tools}
%TODO

ping
nslookup
nmap
IP SLA
show (cisco)
ssh
dhcpcheck
ntpq
iperf
tracepath
\newpage
\section{Device Tests}

\subsection{Check Username}
\subsubsection{Beschreibung}
Mit diesem Test kann man überprüfen, ob alle benötigten Benutzer vorhanden sind.
\subsubsection{Tools}
- Verbindung über SSH auf Cisco UI / Linux Shell
\subsubsection{Devices}
- Cisco Switch/Router
\subsubsection{Commands}
Cisco: sh run | i username
\subsubsection{Parameter}
Keine
\subsubsection{Ausgabe}
username admin privilege 15 secret 5 1G8E3yyKZWRKXRfLPS9Ncq/hxA/
\subsubsection{Bemerkungen}
Keine

\subsection{Check Version}
\subsubsection{Beschreibung}
Durch diesen Test kann man den Versionstand von Switches und Router Testen.
\subsubsection{Tools}
- Verbindung über SSH auf Cisco UI / Linux Shell
\subsubsection{Devices}
- Cisco Switch/Router
\subsubsection{Commands}
Cisco: sh version
\subsubsection{Parameter}
Keine
\subsubsection{Ausgabe}
...\newline
...\newline
Switch Ports Model              SW Version            SW Image\newline
------ ----- -----              ----------            ----------\newline
*    1 26    WS-C2960-24TC-L    15.0(2)SE8            C2960-LANBASEK9-M
\subsubsection{Bemerkungen}
Keine

\newpage
\section{Topology Tests}

\subsection{Spanning-Tree / Port State}
\subsubsection{Beschreibung}
State eines Interfaces auf jedem VLAN.
\subsubsection{Tools}
- Verbindung über SSH auf Cisco UI / Linux Shell
\subsubsection{Devices}
- Cisco Switch/Router
\subsubsection{Commands}
show spanning-tree interface <Interface> state
\subsubsection{Parameter}
Interface: Interface auf dem Switch/Router
\subsubsection{Ausgabe}
VLAN0020            blocking
\subsubsection{Bemerkungen}
Keine

\subsection{Spanning-Tree / mst}
\subsubsection{Beschreibung}
Ist MST aktiviert oder nicht?
\subsubsection{Tools}
- Verbindung über SSH auf Cisco UI / Linux Shell
\subsubsection{Devices}
- Cisco Switch/Router
\subsubsection{Commands}
show spanning-tree mst
\subsubsection{Parameter}
Keine
\subsubsection{Ausgabe}
Switch is not in mst mode
\subsubsection{Bemerkungen}
Keine



\subsection{Spanning-Tree / mst}
\subsubsection{Beschreibung}
Testet, ob die richtigen Interface geblockt sind.
\subsubsection{Tools}
- Verbindung über SSH auf Cisco UI / Linux Shell
\subsubsection{Devices}
- Cisco Switch/Router
\subsubsection{Commands}
sh spanning-tree blockedports
\subsubsection{Parameter}
Keine
\subsubsection{Ausgabe}
Name                 Blocked Interfaces List\newline
-------------------- ------------------------------------\newline
VLAN0020             Fa0/20\newline
\subsubsection{Bemerkungen}
Keine


\subsection{Spanning-Tree / UplinkFast}
\subsubsection{Beschreibung}
Ist UplinkFast aktiviert oder nicht?
\subsubsection{Tools}
- Verbindung über SSH auf Cisco UI / Linux Shell
\subsubsection{Devices}
- Cisco Switch/Router
\subsubsection{Commands}
show spanning-tree portfast
\subsubsection{Parameter}
Keine
\subsubsection{Ausgabe}
UplinkFast is disabled
\subsubsection{Bemerkungen}
Keine


\subsection{Spanning-Tree / Portfast}
\subsubsection{Beschreibung}
In diesem Test kann jedes Interface auf Portfast getestet werden.
\subsubsection{Tools}
- Verbindung über SSH auf Cisco UI / Linux Shell
\subsubsection{Devices}
- Cisco Switch/Router
\subsubsection{Commands}
show spanning-tree interface <Interface> state
\subsubsection{Parameter}
Interface: Interface auf dem Switch/Router
\subsubsection{Ausgabe}
VLAN0020            enabled
\subsubsection{Bemerkungen}
Keine


\subsection{Spanning-Tree / Interface Settings}
\subsubsection{Beschreibung}
In diesem Testszenario testet man alle VLAN-Einstellungen eines Interfaces.
\subsubsection{Tools}
- Verbindung über SSH auf Cisco UI / Linux Shell
\subsubsection{Devices}
- Cisco Switch/Router
\subsubsection{Commands}
show spanning-tree interface <Interface>
\subsubsection{Parameter}
Interface: Interface auf dem Switch/Router
\subsubsection{Ausgabe}
Vlan                Role Sts Cost      Prio.Nbr Type\newline
------------------- ---- --- --------- -------- --------------------------------\newline
VLAN0020            Desg FWD 19        128.21   P2p Edge\newline

Vlan                Role Sts Cost      Prio.Nbr Type\newline
------------------- ---- --- --------- -------- --------------------------------\newline
VLAN0001            Root FWD 19        128.5    P2p\newline
VLAN0010            Root FWD 19        128.5    P2p\newline
\subsubsection{Bemerkungen}
Keine




\subsection{Spanning-Tree / Root für ein VLAN}
\subsubsection{Beschreibung}
Um den Root eines VLANs zu testen kann dieser Test verwendet werden.
\subsubsection{Tools}
- Verbindung über SSH auf Cisco UI / Linux Shell
\subsubsection{Devices}
- Cisco Switch/Router
\subsubsection{Commands}
show spanning-tree vlan <VLAN-ID> root port
\subsubsection{Parameter}
VLAN-ID: Die ID des gewünschten VLANs.
\subsubsection{Ausgabe}
VLAN0020         FastEthernet0/21
\subsubsection{Bemerkungen}
Keine


\subsection{Spanning-Tree / Alle Rootports}
\subsubsection{Beschreibung}
Test über alle VLANs, ob die richtigen Interface als Rootport gesetzt sind.
\subsubsection{Tools}
- Verbindung über SSH auf Cisco UI / Linux Shell
\subsubsection{Devices}
- Cisco Switch/Router
\subsubsection{Commands}
show spanning-tree root
\subsubsection{Parameter}
VLAN-ID: Die ID des gewünschten VLANs.
\subsubsection{Ausgabe}
Vlan                   Root ID          Cost    Time  Age Dly  Root Port\newline
---------------- -------------------- --------- ----- --- ---  ------------\newline
VLAN0001         32769 001a.2fd4.4000        19    2   20  15  Fa0/5\newline
VLAN0010          8202 001a.2fd4.4000        19    2   20  15  Fa0/5\newline
VLAN0011          8203 001a.2fd4.4000        19    2   20  15  Fa0/5\newline
\subsubsection{Bemerkungen}
Keine



\subsection{CDP Neighbor Check}
\subsubsection{Beschreibung}
Dieser Test überprüft, ob alle CDP Neighbor vorhanden sind.
\subsubsection{Tools}
- Verbindung über SSH auf Cisco UI / Linux Shell
\subsubsection{Devices}
- Cisco Switch/Router
\subsubsection{Commands}
show cdp neighbors
\subsubsection{Parameter}
Keine
\subsubsection{Ausgabe}
Capability Codes: R - Router, T - Trans Bridge, B - Source Route Bridge\newline
                  S - Switch, H - Host, I - IGMP, r - Repeater, P - Phone,\newline
                  D - Remote, C - CVTA, M - Two-port Mac Relay\newline

Device ID        Local Intrfce     Holdtme    Capability  Platform  Port ID\newline
HQ_AS7           Fas 0/14          161             R S I  WS-C2960- Fas 0/21\newline
HQ_AS8           Fas 0/15          172              S I   WS-C2960- Fas 0/20\newline
\subsubsection{Bemerkungen}
Keine





\subsection{VLAN Check}
\subsubsection{Beschreibung}
Überprüft, ob die Interface zum richtigen VLAN zugewiesen sind.
\subsubsection{Tools}
- Verbindung über SSH auf Cisco UI / Linux Shell
\subsubsection{Devices}
- Cisco Switch/Router
\subsubsection{Commands}
show vlan id <VLAN-ID>
\subsubsection{Parameter}
Keine
\subsubsection{Ausgabe}
Capability Codes: R - Router, T - Trans Bridge, B - Source Route Bridge\newline
                  S - Switch, H - Host, I - IGMP, r - Repeater, P - Phone,\newline
                  D - Remote, C - CVTA, M - Two-port Mac Relay\newline

Device ID        Local Intrfce     Holdtme    Capability  Platform  Port ID\newline
HQ_AS7           Fas 0/14          161             R S I  WS-C2960- Fas 0/21\newline
HQ_AS8           Fas 0/15          172              S I   WS-C2960- Fas 0/20\newline
\subsubsection{Bemerkungen}
Keine



































\newpage
\section{Traffic Tests}
\subsection{Connectivity}
\subsubsection{Beschreibung}
In diesem Testszenario soll die Connectivity mittels ICMP ping Befehl getestet werden.
\subsubsection{Tools}
- Verbindung über SSH auf Cisco UI / Linux Shell
- ping
- ip sla

\subsubsection{Devices}
- Cisco Switch/Router
- Linux OS
\subsubsection{Commands}
Cisco: "ping <destinationIP> [parameter]"
Linux: "ping -c 1 <destinationIP>"

Cisco ip sla: 	"ip sla 1"
			 	"icmp-echo <destinationIP>"
			  	"request-data-size <0-16384>"
			  	"ip sla schedule 1 start-time now life 0"
			  	"sh ip sla statistics 1"

\subsubsection{Parameter}
- Zwingend Destination, Entry Number (ip sla), optional Source Interface
\subsubsection{Ausgabe}
Cisco ping: "Succes rate is x percent (x/5)"
Linux ping: "x packets transmitted, x received, ..."

Cisco ip sla: "IPSLA Operation id, Latest RTT, start time, return code, number of successes, number of failures,..."

\subsubsection{Bemerkungen}
Mit IP SLA könnte man das Ergebnis via XML erhalten
"show ip sla statistics 1 | format"


\subsection{NTP-Response}
\subsubsection{Beschreibung}
Um die Funktionalität des NTP Servers zu testen, kann man diesen Test verwenden.
\subsubsection{Tools}
- ntpq
\subsubsection{Devices}
- Linux OS
\subsubsection{Commands}
ntpq -p <ntp-server>
\subsubsection{Parameter}
- IP-Adresse oder Hostname des NTP-Servers
\subsubsection{Ausgabe}
     remote           refid      st t when poll reach   delay   offset  jitter\newline
==============================================================================\newline
-soho.solnet.ch  162.23.41.56     2 s  494  512  357    1.313   -0.367   0.233\newline
-bryan.solnet.ch 192.36.143.151   2 s  196  512  377    1.263   -3.990   0.175\newline
 wernerstats.sol 212.101.4.252    2 s 792d   64    0    0.000    0.000   0.000\newline \subsubsection{Bemerkungen}
Keine


\subsection{Webservice-Response}
\subsubsection{Beschreibung}
Mit diesem Test kann die Erreichbarkeit eines Webservers getestet werden.
\subsubsection{Tools}
- curl
\subsubsection{Devices}
- Linux OS
\subsubsection{Commands}
curl -Is <webserver> | head -n 1
\subsubsection{Parameter}
- IP-Adresse oder Hostname des NTP-Servers
\subsubsection{Ausgabe}
HTTP/1.1 200 OK
\subsubsection{Bemerkungen}
Keine

\subsection{Portscan}
\subsubsection{Beschreibung}
Dieses Testszenario gibt zurück, ob ein Port erreichbar ist.
\subsubsection{Tools}
- nmap
\subsubsection{Devices}
- Linux OS
\subsubsection{Commands}
nmap -p<port> <host>
nmap -p<start-ende> <host>
nmap -p<port1, port4> <host>
\subsubsection{Parameter}
- port: Einzelner Port
- start-ende: Portbereich
- port1, port4: Mehrere Ports
- IP-Adresse oder Hostname des NTP-Servers
\subsubsection{Ausgabe}
Starting Nmap 6.47 ( http://nmap.org ) at 2016-10-12 16:21 W. Europe Daylight Time\newline
Nmap scan report for host057.lab.ins.hsr.ch (152.96.9.57)\newline
Host is up (0.00s latency).\newline
PORT    STATE SERVICE\newline
135/tcp open  msrpc\newline
MAC Address: 00:19:99:B3:37:39 (Fujitsu Technology Solutions)\newline
Nmap done: 1 IP address (1 host up) scanned in 1.08 seconds\newline
\subsubsection{Bemerkungen}
In manchen Unternehmen wird ein Portscan als Attacke gewertet. Man sollte dies vorher unbediengt abklären.
Bei Firewalls sollte man vorsichtig sein, denn gewisse Firewalls erkennen Portscans und blocken danach den Verkehr. So kann man vom Netz ausgeschlossen werden.



\newpage
\section{Network Service Tests}
\subsection{DHCP-Response}
\subsubsection{Beschreibung}
In diesem Testszenario wird der DHCP-Dienst auf Funktionalität getestet.
\subsubsection{Tools}
- Verbindung über SSH auf Cisco UI / Linux Shell
- IP SLA
- dhcpcheck
\subsubsection{Devices}
- Cisco OS
- Linux OS
\subsubsection{Commands}
\paragraph{Cisco}
ip sla 1
dhcp <DHCP-Server>
ip sla schedule 1 start-time now life 0
sh ip sla stat 1
no ip sla 1
			
\paragraph{Linux}
%TODO
\subsubsection{Parameter}
- DHCP-Server: IP-Adresse oder Hostname des DHCP-Diestes
\subsubsection{Ausgabe}
\paragraph{Cisco}
Round Trip Time (RTT) for       Index 1
Type of operation: dhcp
Latest operation start time: *04:07:33.884 UTC Mon Mar 1 1993
Latest operation return code: OK
Number of successes: 1
Number of failures: 0
Operation time to live: 0
\paragraph{Linux}
%TODO
\subsubsection{Bemerkungen}
Keine


\subsection{DNS-Response}
\subsubsection{Beschreibung}
Mit diesem Test kann man einen DNS-Server testen, ob er erreichbar ist und korrekt funktioniert.
\subsubsection{Tools}
- Verbindung über SSH auf Cisco UI / Linux Shell
- IP SLA
- nslookup
\subsubsection{Devices}
- Linux OS
\subsubsection{Commands}
\paragraph{Cisco}
ip sla 1
dns <hostname> name-server <nameserver>
tag DNS-Test
ip sla schedule 1 start-time now
sh ip sla stat 1
no ip sla 1
			
\paragraph{Linux}
nslookup <hostname> [type]
\subsubsection{Parameter}
- hostname: Ein Hostname den man auflösen möchte.
- nameserver: Nameserver, welche die Abfrage erledigen soll.
- type: Typ der Abfrage. Z. B. mx, ns, uws.
\subsubsection{Ausgabe}
\paragraph{Cisco}
Round Trip Time (RTT) for       Index 1
Type of operation: dns
Latest operation start time: *04:07:33.884 UTC Mon Mar 1 1993
Latest operation return code: OK
Number of successes: 1
Number of failures: 0
Operation time to live: 0
\paragraph{Linux}
%TODO
\subsubsection{Bemerkungen}
Keine




\newpage
\section{Ausgabeformate}
\subsection{XML}
\subsection{JSON}
\subsection{RAW}

\end{document}

